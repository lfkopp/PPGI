
% Default to the notebook output style

    


% Inherit from the specified cell style.




    
\documentclass[11pt]{article}

    
    
    \usepackage[T1]{fontenc}
    % Nicer default font (+ math font) than Computer Modern for most use cases
    \usepackage{mathpazo}

    % Basic figure setup, for now with no caption control since it's done
    % automatically by Pandoc (which extracts ![](path) syntax from Markdown).
    \usepackage{graphicx}
    % We will generate all images so they have a width \maxwidth. This means
    % that they will get their normal width if they fit onto the page, but
    % are scaled down if they would overflow the margins.
    \makeatletter
    \def\maxwidth{\ifdim\Gin@nat@width>\linewidth\linewidth
    \else\Gin@nat@width\fi}
    \makeatother
    \let\Oldincludegraphics\includegraphics
    % Set max figure width to be 80% of text width, for now hardcoded.
    \renewcommand{\includegraphics}[1]{\Oldincludegraphics[width=.8\maxwidth]{#1}}
    % Ensure that by default, figures have no caption (until we provide a
    % proper Figure object with a Caption API and a way to capture that
    % in the conversion process - todo).
    \usepackage{caption}
    \DeclareCaptionLabelFormat{nolabel}{}
    \captionsetup{labelformat=nolabel}

    \usepackage{adjustbox} % Used to constrain images to a maximum size 
    \usepackage{xcolor} % Allow colors to be defined
    \usepackage{enumerate} % Needed for markdown enumerations to work
    \usepackage{geometry} % Used to adjust the document margins
    \usepackage{amsmath} % Equations
    \usepackage{amssymb} % Equations
    \usepackage{textcomp} % defines textquotesingle
    % Hack from http://tex.stackexchange.com/a/47451/13684:
    \AtBeginDocument{%
        \def\PYZsq{\textquotesingle}% Upright quotes in Pygmentized code
    }
    \usepackage{upquote} % Upright quotes for verbatim code
    \usepackage{eurosym} % defines \euro
    \usepackage[mathletters]{ucs} % Extended unicode (utf-8) support
    \usepackage[utf8x]{inputenc} % Allow utf-8 characters in the tex document
    \usepackage{fancyvrb} % verbatim replacement that allows latex
    \usepackage{grffile} % extends the file name processing of package graphics 
                         % to support a larger range 
    % The hyperref package gives us a pdf with properly built
    % internal navigation ('pdf bookmarks' for the table of contents,
    % internal cross-reference links, web links for URLs, etc.)
    \usepackage{hyperref}
    \usepackage{longtable} % longtable support required by pandoc >1.10
    \usepackage{booktabs}  % table support for pandoc > 1.12.2
    \usepackage[inline]{enumitem} % IRkernel/repr support (it uses the enumerate* environment)
    \usepackage[normalem]{ulem} % ulem is needed to support strikethroughs (\sout)
                                % normalem makes italics be italics, not underlines
    

    
    
    % Colors for the hyperref package
    \definecolor{urlcolor}{rgb}{0,.145,.698}
    \definecolor{linkcolor}{rgb}{.71,0.21,0.01}
    \definecolor{citecolor}{rgb}{.12,.54,.11}

    % ANSI colors
    \definecolor{ansi-black}{HTML}{3E424D}
    \definecolor{ansi-black-intense}{HTML}{282C36}
    \definecolor{ansi-red}{HTML}{E75C58}
    \definecolor{ansi-red-intense}{HTML}{B22B31}
    \definecolor{ansi-green}{HTML}{00A250}
    \definecolor{ansi-green-intense}{HTML}{007427}
    \definecolor{ansi-yellow}{HTML}{DDB62B}
    \definecolor{ansi-yellow-intense}{HTML}{B27D12}
    \definecolor{ansi-blue}{HTML}{208FFB}
    \definecolor{ansi-blue-intense}{HTML}{0065CA}
    \definecolor{ansi-magenta}{HTML}{D160C4}
    \definecolor{ansi-magenta-intense}{HTML}{A03196}
    \definecolor{ansi-cyan}{HTML}{60C6C8}
    \definecolor{ansi-cyan-intense}{HTML}{258F8F}
    \definecolor{ansi-white}{HTML}{C5C1B4}
    \definecolor{ansi-white-intense}{HTML}{A1A6B2}

    % commands and environments needed by pandoc snippets
    % extracted from the output of `pandoc -s`
    \providecommand{\tightlist}{%
      \setlength{\itemsep}{0pt}\setlength{\parskip}{0pt}}
    \DefineVerbatimEnvironment{Highlighting}{Verbatim}{commandchars=\\\{\}}
    % Add ',fontsize=\small' for more characters per line
    \newenvironment{Shaded}{}{}
    \newcommand{\KeywordTok}[1]{\textcolor[rgb]{0.00,0.44,0.13}{\textbf{{#1}}}}
    \newcommand{\DataTypeTok}[1]{\textcolor[rgb]{0.56,0.13,0.00}{{#1}}}
    \newcommand{\DecValTok}[1]{\textcolor[rgb]{0.25,0.63,0.44}{{#1}}}
    \newcommand{\BaseNTok}[1]{\textcolor[rgb]{0.25,0.63,0.44}{{#1}}}
    \newcommand{\FloatTok}[1]{\textcolor[rgb]{0.25,0.63,0.44}{{#1}}}
    \newcommand{\CharTok}[1]{\textcolor[rgb]{0.25,0.44,0.63}{{#1}}}
    \newcommand{\StringTok}[1]{\textcolor[rgb]{0.25,0.44,0.63}{{#1}}}
    \newcommand{\CommentTok}[1]{\textcolor[rgb]{0.38,0.63,0.69}{\textit{{#1}}}}
    \newcommand{\OtherTok}[1]{\textcolor[rgb]{0.00,0.44,0.13}{{#1}}}
    \newcommand{\AlertTok}[1]{\textcolor[rgb]{1.00,0.00,0.00}{\textbf{{#1}}}}
    \newcommand{\FunctionTok}[1]{\textcolor[rgb]{0.02,0.16,0.49}{{#1}}}
    \newcommand{\RegionMarkerTok}[1]{{#1}}
    \newcommand{\ErrorTok}[1]{\textcolor[rgb]{1.00,0.00,0.00}{\textbf{{#1}}}}
    \newcommand{\NormalTok}[1]{{#1}}
    
    % Additional commands for more recent versions of Pandoc
    \newcommand{\ConstantTok}[1]{\textcolor[rgb]{0.53,0.00,0.00}{{#1}}}
    \newcommand{\SpecialCharTok}[1]{\textcolor[rgb]{0.25,0.44,0.63}{{#1}}}
    \newcommand{\VerbatimStringTok}[1]{\textcolor[rgb]{0.25,0.44,0.63}{{#1}}}
    \newcommand{\SpecialStringTok}[1]{\textcolor[rgb]{0.73,0.40,0.53}{{#1}}}
    \newcommand{\ImportTok}[1]{{#1}}
    \newcommand{\DocumentationTok}[1]{\textcolor[rgb]{0.73,0.13,0.13}{\textit{{#1}}}}
    \newcommand{\AnnotationTok}[1]{\textcolor[rgb]{0.38,0.63,0.69}{\textbf{\textit{{#1}}}}}
    \newcommand{\CommentVarTok}[1]{\textcolor[rgb]{0.38,0.63,0.69}{\textbf{\textit{{#1}}}}}
    \newcommand{\VariableTok}[1]{\textcolor[rgb]{0.10,0.09,0.49}{{#1}}}
    \newcommand{\ControlFlowTok}[1]{\textcolor[rgb]{0.00,0.44,0.13}{\textbf{{#1}}}}
    \newcommand{\OperatorTok}[1]{\textcolor[rgb]{0.40,0.40,0.40}{{#1}}}
    \newcommand{\BuiltInTok}[1]{{#1}}
    \newcommand{\ExtensionTok}[1]{{#1}}
    \newcommand{\PreprocessorTok}[1]{\textcolor[rgb]{0.74,0.48,0.00}{{#1}}}
    \newcommand{\AttributeTok}[1]{\textcolor[rgb]{0.49,0.56,0.16}{{#1}}}
    \newcommand{\InformationTok}[1]{\textcolor[rgb]{0.38,0.63,0.69}{\textbf{\textit{{#1}}}}}
    \newcommand{\WarningTok}[1]{\textcolor[rgb]{0.38,0.63,0.69}{\textbf{\textit{{#1}}}}}
    
    
    % Define a nice break command that doesn't care if a line doesn't already
    % exist.
    \def\br{\hspace*{\fill} \\* }
    % Math Jax compatability definitions
    \def\gt{>}
    \def\lt{<}
    % Document parameters
    \title{lista\_4}
    
    
    

    % Pygments definitions
    
\makeatletter
\def\PY@reset{\let\PY@it=\relax \let\PY@bf=\relax%
    \let\PY@ul=\relax \let\PY@tc=\relax%
    \let\PY@bc=\relax \let\PY@ff=\relax}
\def\PY@tok#1{\csname PY@tok@#1\endcsname}
\def\PY@toks#1+{\ifx\relax#1\empty\else%
    \PY@tok{#1}\expandafter\PY@toks\fi}
\def\PY@do#1{\PY@bc{\PY@tc{\PY@ul{%
    \PY@it{\PY@bf{\PY@ff{#1}}}}}}}
\def\PY#1#2{\PY@reset\PY@toks#1+\relax+\PY@do{#2}}

\expandafter\def\csname PY@tok@w\endcsname{\def\PY@tc##1{\textcolor[rgb]{0.73,0.73,0.73}{##1}}}
\expandafter\def\csname PY@tok@c\endcsname{\let\PY@it=\textit\def\PY@tc##1{\textcolor[rgb]{0.25,0.50,0.50}{##1}}}
\expandafter\def\csname PY@tok@cp\endcsname{\def\PY@tc##1{\textcolor[rgb]{0.74,0.48,0.00}{##1}}}
\expandafter\def\csname PY@tok@k\endcsname{\let\PY@bf=\textbf\def\PY@tc##1{\textcolor[rgb]{0.00,0.50,0.00}{##1}}}
\expandafter\def\csname PY@tok@kp\endcsname{\def\PY@tc##1{\textcolor[rgb]{0.00,0.50,0.00}{##1}}}
\expandafter\def\csname PY@tok@kt\endcsname{\def\PY@tc##1{\textcolor[rgb]{0.69,0.00,0.25}{##1}}}
\expandafter\def\csname PY@tok@o\endcsname{\def\PY@tc##1{\textcolor[rgb]{0.40,0.40,0.40}{##1}}}
\expandafter\def\csname PY@tok@ow\endcsname{\let\PY@bf=\textbf\def\PY@tc##1{\textcolor[rgb]{0.67,0.13,1.00}{##1}}}
\expandafter\def\csname PY@tok@nb\endcsname{\def\PY@tc##1{\textcolor[rgb]{0.00,0.50,0.00}{##1}}}
\expandafter\def\csname PY@tok@nf\endcsname{\def\PY@tc##1{\textcolor[rgb]{0.00,0.00,1.00}{##1}}}
\expandafter\def\csname PY@tok@nc\endcsname{\let\PY@bf=\textbf\def\PY@tc##1{\textcolor[rgb]{0.00,0.00,1.00}{##1}}}
\expandafter\def\csname PY@tok@nn\endcsname{\let\PY@bf=\textbf\def\PY@tc##1{\textcolor[rgb]{0.00,0.00,1.00}{##1}}}
\expandafter\def\csname PY@tok@ne\endcsname{\let\PY@bf=\textbf\def\PY@tc##1{\textcolor[rgb]{0.82,0.25,0.23}{##1}}}
\expandafter\def\csname PY@tok@nv\endcsname{\def\PY@tc##1{\textcolor[rgb]{0.10,0.09,0.49}{##1}}}
\expandafter\def\csname PY@tok@no\endcsname{\def\PY@tc##1{\textcolor[rgb]{0.53,0.00,0.00}{##1}}}
\expandafter\def\csname PY@tok@nl\endcsname{\def\PY@tc##1{\textcolor[rgb]{0.63,0.63,0.00}{##1}}}
\expandafter\def\csname PY@tok@ni\endcsname{\let\PY@bf=\textbf\def\PY@tc##1{\textcolor[rgb]{0.60,0.60,0.60}{##1}}}
\expandafter\def\csname PY@tok@na\endcsname{\def\PY@tc##1{\textcolor[rgb]{0.49,0.56,0.16}{##1}}}
\expandafter\def\csname PY@tok@nt\endcsname{\let\PY@bf=\textbf\def\PY@tc##1{\textcolor[rgb]{0.00,0.50,0.00}{##1}}}
\expandafter\def\csname PY@tok@nd\endcsname{\def\PY@tc##1{\textcolor[rgb]{0.67,0.13,1.00}{##1}}}
\expandafter\def\csname PY@tok@s\endcsname{\def\PY@tc##1{\textcolor[rgb]{0.73,0.13,0.13}{##1}}}
\expandafter\def\csname PY@tok@sd\endcsname{\let\PY@it=\textit\def\PY@tc##1{\textcolor[rgb]{0.73,0.13,0.13}{##1}}}
\expandafter\def\csname PY@tok@si\endcsname{\let\PY@bf=\textbf\def\PY@tc##1{\textcolor[rgb]{0.73,0.40,0.53}{##1}}}
\expandafter\def\csname PY@tok@se\endcsname{\let\PY@bf=\textbf\def\PY@tc##1{\textcolor[rgb]{0.73,0.40,0.13}{##1}}}
\expandafter\def\csname PY@tok@sr\endcsname{\def\PY@tc##1{\textcolor[rgb]{0.73,0.40,0.53}{##1}}}
\expandafter\def\csname PY@tok@ss\endcsname{\def\PY@tc##1{\textcolor[rgb]{0.10,0.09,0.49}{##1}}}
\expandafter\def\csname PY@tok@sx\endcsname{\def\PY@tc##1{\textcolor[rgb]{0.00,0.50,0.00}{##1}}}
\expandafter\def\csname PY@tok@m\endcsname{\def\PY@tc##1{\textcolor[rgb]{0.40,0.40,0.40}{##1}}}
\expandafter\def\csname PY@tok@gh\endcsname{\let\PY@bf=\textbf\def\PY@tc##1{\textcolor[rgb]{0.00,0.00,0.50}{##1}}}
\expandafter\def\csname PY@tok@gu\endcsname{\let\PY@bf=\textbf\def\PY@tc##1{\textcolor[rgb]{0.50,0.00,0.50}{##1}}}
\expandafter\def\csname PY@tok@gd\endcsname{\def\PY@tc##1{\textcolor[rgb]{0.63,0.00,0.00}{##1}}}
\expandafter\def\csname PY@tok@gi\endcsname{\def\PY@tc##1{\textcolor[rgb]{0.00,0.63,0.00}{##1}}}
\expandafter\def\csname PY@tok@gr\endcsname{\def\PY@tc##1{\textcolor[rgb]{1.00,0.00,0.00}{##1}}}
\expandafter\def\csname PY@tok@ge\endcsname{\let\PY@it=\textit}
\expandafter\def\csname PY@tok@gs\endcsname{\let\PY@bf=\textbf}
\expandafter\def\csname PY@tok@gp\endcsname{\let\PY@bf=\textbf\def\PY@tc##1{\textcolor[rgb]{0.00,0.00,0.50}{##1}}}
\expandafter\def\csname PY@tok@go\endcsname{\def\PY@tc##1{\textcolor[rgb]{0.53,0.53,0.53}{##1}}}
\expandafter\def\csname PY@tok@gt\endcsname{\def\PY@tc##1{\textcolor[rgb]{0.00,0.27,0.87}{##1}}}
\expandafter\def\csname PY@tok@err\endcsname{\def\PY@bc##1{\setlength{\fboxsep}{0pt}\fcolorbox[rgb]{1.00,0.00,0.00}{1,1,1}{\strut ##1}}}
\expandafter\def\csname PY@tok@kc\endcsname{\let\PY@bf=\textbf\def\PY@tc##1{\textcolor[rgb]{0.00,0.50,0.00}{##1}}}
\expandafter\def\csname PY@tok@kd\endcsname{\let\PY@bf=\textbf\def\PY@tc##1{\textcolor[rgb]{0.00,0.50,0.00}{##1}}}
\expandafter\def\csname PY@tok@kn\endcsname{\let\PY@bf=\textbf\def\PY@tc##1{\textcolor[rgb]{0.00,0.50,0.00}{##1}}}
\expandafter\def\csname PY@tok@kr\endcsname{\let\PY@bf=\textbf\def\PY@tc##1{\textcolor[rgb]{0.00,0.50,0.00}{##1}}}
\expandafter\def\csname PY@tok@bp\endcsname{\def\PY@tc##1{\textcolor[rgb]{0.00,0.50,0.00}{##1}}}
\expandafter\def\csname PY@tok@fm\endcsname{\def\PY@tc##1{\textcolor[rgb]{0.00,0.00,1.00}{##1}}}
\expandafter\def\csname PY@tok@vc\endcsname{\def\PY@tc##1{\textcolor[rgb]{0.10,0.09,0.49}{##1}}}
\expandafter\def\csname PY@tok@vg\endcsname{\def\PY@tc##1{\textcolor[rgb]{0.10,0.09,0.49}{##1}}}
\expandafter\def\csname PY@tok@vi\endcsname{\def\PY@tc##1{\textcolor[rgb]{0.10,0.09,0.49}{##1}}}
\expandafter\def\csname PY@tok@vm\endcsname{\def\PY@tc##1{\textcolor[rgb]{0.10,0.09,0.49}{##1}}}
\expandafter\def\csname PY@tok@sa\endcsname{\def\PY@tc##1{\textcolor[rgb]{0.73,0.13,0.13}{##1}}}
\expandafter\def\csname PY@tok@sb\endcsname{\def\PY@tc##1{\textcolor[rgb]{0.73,0.13,0.13}{##1}}}
\expandafter\def\csname PY@tok@sc\endcsname{\def\PY@tc##1{\textcolor[rgb]{0.73,0.13,0.13}{##1}}}
\expandafter\def\csname PY@tok@dl\endcsname{\def\PY@tc##1{\textcolor[rgb]{0.73,0.13,0.13}{##1}}}
\expandafter\def\csname PY@tok@s2\endcsname{\def\PY@tc##1{\textcolor[rgb]{0.73,0.13,0.13}{##1}}}
\expandafter\def\csname PY@tok@sh\endcsname{\def\PY@tc##1{\textcolor[rgb]{0.73,0.13,0.13}{##1}}}
\expandafter\def\csname PY@tok@s1\endcsname{\def\PY@tc##1{\textcolor[rgb]{0.73,0.13,0.13}{##1}}}
\expandafter\def\csname PY@tok@mb\endcsname{\def\PY@tc##1{\textcolor[rgb]{0.40,0.40,0.40}{##1}}}
\expandafter\def\csname PY@tok@mf\endcsname{\def\PY@tc##1{\textcolor[rgb]{0.40,0.40,0.40}{##1}}}
\expandafter\def\csname PY@tok@mh\endcsname{\def\PY@tc##1{\textcolor[rgb]{0.40,0.40,0.40}{##1}}}
\expandafter\def\csname PY@tok@mi\endcsname{\def\PY@tc##1{\textcolor[rgb]{0.40,0.40,0.40}{##1}}}
\expandafter\def\csname PY@tok@il\endcsname{\def\PY@tc##1{\textcolor[rgb]{0.40,0.40,0.40}{##1}}}
\expandafter\def\csname PY@tok@mo\endcsname{\def\PY@tc##1{\textcolor[rgb]{0.40,0.40,0.40}{##1}}}
\expandafter\def\csname PY@tok@ch\endcsname{\let\PY@it=\textit\def\PY@tc##1{\textcolor[rgb]{0.25,0.50,0.50}{##1}}}
\expandafter\def\csname PY@tok@cm\endcsname{\let\PY@it=\textit\def\PY@tc##1{\textcolor[rgb]{0.25,0.50,0.50}{##1}}}
\expandafter\def\csname PY@tok@cpf\endcsname{\let\PY@it=\textit\def\PY@tc##1{\textcolor[rgb]{0.25,0.50,0.50}{##1}}}
\expandafter\def\csname PY@tok@c1\endcsname{\let\PY@it=\textit\def\PY@tc##1{\textcolor[rgb]{0.25,0.50,0.50}{##1}}}
\expandafter\def\csname PY@tok@cs\endcsname{\let\PY@it=\textit\def\PY@tc##1{\textcolor[rgb]{0.25,0.50,0.50}{##1}}}

\def\PYZbs{\char`\\}
\def\PYZus{\char`\_}
\def\PYZob{\char`\{}
\def\PYZcb{\char`\}}
\def\PYZca{\char`\^}
\def\PYZam{\char`\&}
\def\PYZlt{\char`\<}
\def\PYZgt{\char`\>}
\def\PYZsh{\char`\#}
\def\PYZpc{\char`\%}
\def\PYZdl{\char`\$}
\def\PYZhy{\char`\-}
\def\PYZsq{\char`\'}
\def\PYZdq{\char`\"}
\def\PYZti{\char`\~}
% for compatibility with earlier versions
\def\PYZat{@}
\def\PYZlb{[}
\def\PYZrb{]}
\makeatother


    % Exact colors from NB
    \definecolor{incolor}{rgb}{0.0, 0.0, 0.5}
    \definecolor{outcolor}{rgb}{0.545, 0.0, 0.0}



    
    % Prevent overflowing lines due to hard-to-break entities
    \sloppy 
    % Setup hyperref package
    \hypersetup{
      breaklinks=true,  % so long urls are correctly broken across lines
      colorlinks=true,
      urlcolor=urlcolor,
      linkcolor=linkcolor,
      citecolor=citecolor,
      }
    % Slightly bigger margins than the latex defaults
    
    \geometry{verbose,tmargin=1in,bmargin=1in,lmargin=1in,rmargin=1in}
    
    

    \begin{document}
    
    
    \maketitle
    
    

    
    MAI 103: Análise de Risco // Prof. Eber\\
Lista 03 // Data: 03/07/2018 // Entrega: 10/07/2018

Luis Filipe Kopp\\
Mauro Bastos\\
Brenda Santos\\
Ronilson Pinho

    \begin{Verbatim}[commandchars=\\\{\}]
{\color{incolor}In [{\color{incolor}2}]:} \PY{k+kn}{library}\PY{p}{(}triangle\PY{p}{)}
        \PY{k+kp}{set.seed}\PY{p}{(}\PY{l+m}{1}\PY{p}{)}
\end{Verbatim}


    \begin{Verbatim}[commandchars=\\\{\}]
Warning message:
"package 'triangle' was built under R version 3.3.3"
    \end{Verbatim}

    \begin{enumerate}
\def\labelenumi{\arabic{enumi})}
\tightlist
\item
  A frota de uma empresa de taxi é composta por 20 veículos. Cada uma
  deles consome (40,60,58) litros de gasolina por dia a um custo
  variável de (3.1,4.0,3.8) reais por litro. Crie 3 modelos de risco de
  custo para o gasto diário da empresa de taxi.
\end{enumerate}

1 - simulando gastos individuais para cada um dos 20 taxis da frota

    \begin{Verbatim}[commandchars=\\\{\}]
{\color{incolor}In [{\color{incolor}3}]:} frota1 \PY{o}{\PYZlt{}\PYZhy{}} \PY{k+kt}{c}\PY{p}{(}\PY{p}{)}
        \PY{k+kr}{for} \PY{p}{(}i \PY{k+kr}{in} \PY{l+m}{1}\PY{o}{:}\PY{l+m}{3000}\PY{p}{)}\PY{p}{\PYZob{}}
            litros \PY{o}{\PYZlt{}\PYZhy{}} rtriangle\PY{p}{(}\PY{l+m}{20}\PY{p}{,}\PY{l+m}{40}\PY{p}{,}\PY{l+m}{60}\PY{p}{,}\PY{l+m}{58}\PY{p}{)}
            preco \PY{o}{\PYZlt{}\PYZhy{}} rtriangle\PY{p}{(}\PY{l+m}{20}\PY{p}{,}\PY{l+m}{3.1}\PY{p}{,}\PY{l+m}{4.0}\PY{p}{,}\PY{l+m}{3.8}\PY{p}{)}
            frota1 \PY{o}{\PYZlt{}\PYZhy{}} \PY{k+kt}{c}\PY{p}{(}frota1\PY{p}{,}\PY{k+kp}{sum}\PY{p}{(}litros \PY{o}{*} preco\PY{p}{)}\PY{p}{)}
        \PY{p}{\PYZcb{}}
        \PY{k+kp}{mean}\PY{p}{(}frota1\PY{p}{)}
        hist\PY{p}{(}frota1\PY{p}{)}
\end{Verbatim}


    3825.30102817693

    
    \begin{center}
    \adjustimage{max size={0.9\linewidth}{0.9\paperheight}}{output_3_1.png}
    \end{center}
    { \hspace*{\fill} \\}
    
    2 - simulando o gasto diário de um táxi e usando o TCL

    \begin{Verbatim}[commandchars=\\\{\}]
{\color{incolor}In [{\color{incolor}60}]:} gasto \PY{o}{\PYZlt{}\PYZhy{}} \PY{k+kt}{c}\PY{p}{(}\PY{p}{)}
         \PY{k+kr}{for} \PY{p}{(}i \PY{k+kr}{in} \PY{l+m}{1}\PY{o}{:}\PY{l+m}{1000}\PY{p}{)}\PY{p}{\PYZob{}}
             litros \PY{o}{\PYZlt{}\PYZhy{}} rtriangle\PY{p}{(}\PY{l+m}{1}\PY{p}{,}\PY{l+m}{40}\PY{p}{,}\PY{l+m}{60}\PY{p}{,}\PY{l+m}{58}\PY{p}{)}
             preco \PY{o}{\PYZlt{}\PYZhy{}} rtriangle\PY{p}{(}\PY{l+m}{1}\PY{p}{,}\PY{l+m}{3.1}\PY{p}{,}\PY{l+m}{4.0}\PY{p}{,}\PY{l+m}{3.8}\PY{p}{)}
             gasto \PY{o}{\PYZlt{}\PYZhy{}} \PY{k+kt}{c}\PY{p}{(}gasto\PY{p}{,}\PY{k+kp}{mean}\PY{p}{(}litros \PY{o}{*} preco\PY{p}{)}\PY{p}{)}
         \PY{p}{\PYZcb{}}
         m \PY{o}{\PYZlt{}\PYZhy{}} \PY{k+kp}{mean}\PY{p}{(}\PY{l+m}{20}\PY{o}{*}gasto\PY{p}{)}
         s \PY{o}{\PYZlt{}\PYZhy{}} \PY{k+kp}{sqrt}\PY{p}{(}\PY{l+m}{20}\PY{p}{)}\PY{o}{*}sd\PY{p}{(}gasto\PY{p}{)}
         frota2 \PY{o}{\PYZlt{}\PYZhy{}} rnorm\PY{p}{(}\PY{l+m}{10000}\PY{p}{,}m\PY{p}{,}s\PY{p}{)}
         hist\PY{p}{(}frota2\PY{p}{)}
\end{Verbatim}


    \begin{center}
    \adjustimage{max size={0.9\linewidth}{0.9\paperheight}}{output_5_0.png}
    \end{center}
    { \hspace*{\fill} \\}
    
    \begin{Verbatim}[commandchars=\\\{\}]
{\color{incolor}In [{\color{incolor}44}]:} m\PYZus{}litro \PY{o}{\PYZlt{}\PYZhy{}}\PY{p}{(}\PY{l+m}{40}\PY{l+m}{+60}\PY{l+m}{+58}\PY{p}{)}\PY{o}{/}\PY{l+m}{3}
         m\PYZus{}preco \PY{o}{\PYZlt{}\PYZhy{}} \PY{p}{(}\PY{l+m}{3.1}\PY{l+m}{+4.0}\PY{l+m}{+3.8}\PY{p}{)}\PY{o}{/}\PY{l+m}{3}
         s\PYZus{}litro \PY{o}{\PYZlt{}\PYZhy{}} \PY{k+kp}{sqrt}\PY{p}{(}\PY{p}{(}\PY{l+m}{40}\PY{o}{\PYZca{}}\PY{l+m}{2}\PY{l+m}{+60}\PY{o}{\PYZca{}}\PY{l+m}{2}\PY{l+m}{+58}\PY{o}{\PYZca{}}\PY{l+m}{2}\PY{l+m}{\PYZhy{}40}\PY{o}{*}\PY{l+m}{60}\PY{l+m}{\PYZhy{}40}\PY{o}{*}\PY{l+m}{58}\PY{l+m}{\PYZhy{}60}\PY{o}{*}\PY{l+m}{58}\PY{p}{)}\PY{o}{/}\PY{l+m}{18}\PY{p}{)}
         s\PYZus{}preco \PY{o}{\PYZlt{}\PYZhy{}} \PY{k+kp}{sqrt}\PY{p}{(}\PY{p}{(}\PY{l+m}{3.1}\PY{o}{\PYZca{}}\PY{l+m}{2}\PY{l+m}{+4.0}\PY{o}{\PYZca{}}\PY{l+m}{2}\PY{l+m}{+3.8}\PY{o}{\PYZca{}}\PY{l+m}{2}\PY{l+m}{\PYZhy{}3.1}\PY{o}{*}\PY{l+m}{4.0}\PY{l+m}{\PYZhy{}3.1}\PY{o}{*}\PY{l+m}{3.8}\PY{l+m}{\PYZhy{}4.0}\PY{o}{*}\PY{l+m}{3.8}\PY{p}{)}\PY{o}{/}\PY{l+m}{18}\PY{p}{)}
         
         m \PY{o}{\PYZlt{}\PYZhy{}} m\PYZus{}litro \PY{o}{*} m\PYZus{}preco
         s \PY{o}{\PYZlt{}\PYZhy{}} \PY{k+kp}{sqrt}\PY{p}{(}m\PYZus{}litro\PY{o}{\PYZca{}}\PY{l+m}{2} \PY{o}{*} s\PYZus{}preco\PY{o}{\PYZca{}}\PY{l+m}{2} \PY{o}{+} m\PYZus{}preco\PY{o}{\PYZca{}}\PY{l+m}{2} \PY{o}{*} s\PYZus{}litro\PY{o}{\PYZca{}}\PY{l+m}{2} \PY{o}{+} s\PYZus{}litro\PY{o}{\PYZca{}}\PY{l+m}{2}\PY{o}{*}s\PYZus{}preco\PY{o}{\PYZca{}}\PY{l+m}{2}\PY{p}{)}
         
         frota2b \PY{o}{\PYZlt{}\PYZhy{}}  rnorm\PY{p}{(}\PY{l+m}{10000}\PY{p}{,}\PY{l+m}{20} \PY{o}{*}m\PY{p}{,}\PY{k+kp}{sqrt}\PY{p}{(}\PY{l+m}{20}\PY{p}{)}\PY{o}{*}s\PY{p}{)}
         hist\PY{p}{(}frota2b\PY{p}{)}
         
         
         \PY{c+c1}{\PYZsh{}\PYZsh{} source: doi:10.7151/dmps.1146}
         \PY{c+c1}{\PYZsh{} http://pldml.icm.edu.pl/pldml/element/bwmeta1.element.bwnjournal\PYZhy{}article\PYZhy{}doi\PYZhy{}10\PYZus{}7151\PYZus{}dmps\PYZus{}1146/c/dmps.1146.pdf}
         sc \PY{o}{\PYZlt{}\PYZhy{}} \PY{k+kp}{sqrt}\PY{p}{(}m\PYZus{}litro\PY{o}{\PYZca{}}\PY{l+m}{2} \PY{o}{*} s\PYZus{}preco\PY{o}{\PYZca{}}\PY{l+m}{2} \PY{o}{+} m\PYZus{}preco\PY{o}{\PYZca{}}\PY{l+m}{2} \PY{o}{*} s\PYZus{}litro\PY{o}{\PYZca{}}\PY{l+m}{2}\PY{p}{)}
         frota2c \PY{o}{\PYZlt{}\PYZhy{}}  rnorm\PY{p}{(}\PY{l+m}{50000}\PY{p}{,}\PY{l+m}{20} \PY{o}{*}m\PY{p}{,}\PY{k+kp}{sqrt}\PY{p}{(}\PY{l+m}{20}\PY{p}{)}\PY{o}{*}sc\PY{p}{)}
         hist\PY{p}{(}frota2c\PY{p}{)}
\end{Verbatim}


    \begin{center}
    \adjustimage{max size={0.9\linewidth}{0.9\paperheight}}{output_6_0.png}
    \end{center}
    { \hspace*{\fill} \\}
    
    \begin{center}
    \adjustimage{max size={0.9\linewidth}{0.9\paperheight}}{output_6_1.png}
    \end{center}
    { \hspace*{\fill} \\}
    
    3 - usando a fórmula aproximada para o produto de 2 VAs e aplicando o
TCL

    \begin{Verbatim}[commandchars=\\\{\}]
{\color{incolor}In [{\color{incolor}45}]:} media \PY{o}{\PYZlt{}\PYZhy{}} \PY{k+kt}{c}\PY{p}{(}\PY{p}{)}
         \PY{k+kr}{for} \PY{p}{(}i \PY{k+kr}{in} \PY{l+m}{1}\PY{o}{:}\PY{l+m}{1000}\PY{p}{)}\PY{p}{\PYZob{}}
             litros \PY{o}{\PYZlt{}\PYZhy{}} rtriangle\PY{p}{(}\PY{l+m}{20}\PY{p}{,}\PY{l+m}{40}\PY{p}{,}\PY{l+m}{60}\PY{p}{,}\PY{l+m}{58}\PY{p}{)}
             preco \PY{o}{\PYZlt{}\PYZhy{}} rtriangle\PY{p}{(}\PY{l+m}{20}\PY{p}{,}\PY{l+m}{3.1}\PY{p}{,}\PY{l+m}{4.0}\PY{p}{,}\PY{l+m}{3.8}\PY{p}{)}
             media \PY{o}{\PYZlt{}\PYZhy{}} \PY{k+kt}{c}\PY{p}{(}media\PY{p}{,}\PY{k+kp}{sum}\PY{p}{(}litros \PY{o}{*} preco\PY{p}{)}\PY{p}{)}
         \PY{p}{\PYZcb{}}
         m \PY{o}{\PYZlt{}\PYZhy{}} \PY{k+kp}{mean}\PY{p}{(}media\PY{p}{)}
         s \PY{o}{\PYZlt{}\PYZhy{}} sd\PY{p}{(}media\PY{p}{)}
         frota3 \PY{o}{\PYZlt{}\PYZhy{}} rnorm\PY{p}{(}\PY{l+m}{50000}\PY{p}{,}m\PY{p}{,}s\PY{p}{)}
         hist\PY{p}{(}frota3\PY{p}{)}
\end{Verbatim}


    \begin{center}
    \adjustimage{max size={0.9\linewidth}{0.9\paperheight}}{output_8_0.png}
    \end{center}
    { \hspace*{\fill} \\}
    
    \begin{Verbatim}[commandchars=\\\{\}]
{\color{incolor}In [{\color{incolor}61}]:} plot\PY{p}{(}density\PY{p}{(}frota1\PY{p}{)}\PY{p}{,} col\PY{o}{=}\PY{l+s}{\PYZdq{}}\PY{l+s}{red\PYZdq{}}\PY{p}{)}
         lines\PY{p}{(}density\PY{p}{(}frota2\PY{p}{)}\PY{p}{,} col\PY{o}{=}\PY{l+s}{\PYZdq{}}\PY{l+s}{blue\PYZdq{}}\PY{p}{)}
         lines\PY{p}{(}density\PY{p}{(}frota2b\PY{p}{)}\PY{p}{,} col\PY{o}{=}\PY{l+s}{\PYZdq{}}\PY{l+s}{pink\PYZdq{}}\PY{p}{)}
         lines\PY{p}{(}density\PY{p}{(}frota2c\PY{p}{)}\PY{p}{,} col\PY{o}{=}\PY{l+s}{\PYZdq{}}\PY{l+s}{cyan\PYZdq{}}\PY{p}{)}
         lines\PY{p}{(}density\PY{p}{(}frota3\PY{p}{)}\PY{p}{,} col\PY{o}{=}\PY{l+s}{\PYZdq{}}\PY{l+s}{green\PYZdq{}}\PY{p}{)}
\end{Verbatim}


    \begin{center}
    \adjustimage{max size={0.9\linewidth}{0.9\paperheight}}{output_9_0.png}
    \end{center}
    { \hspace*{\fill} \\}
    
    4 - Compare e discuta os resultados encontrados. Use a função qqnorm
para visualizar a comparação dos resultados.

    em todas as distribuiçõs,perto de zero, as distribuições analisadas
apresentam comportamento quase idêntica ao da normal.

    \begin{Verbatim}[commandchars=\\\{\}]
{\color{incolor}In [{\color{incolor}54}]:} qqnorm\PY{p}{(}frota1\PY{p}{)}
         qqline\PY{p}{(}frota1\PY{p}{,} col\PY{o}{=}\PY{l+s}{\PYZdq{}}\PY{l+s}{red\PYZdq{}}\PY{p}{)}
         qqnorm\PY{p}{(}frota2\PY{p}{)}
         qqline\PY{p}{(}frota2\PY{p}{,} col\PY{o}{=}\PY{l+s}{\PYZdq{}}\PY{l+s}{blue\PYZdq{}}\PY{p}{)}
         qqnorm\PY{p}{(}frota2\PY{p}{)}
         qqline\PY{p}{(}frota2\PY{p}{,} col\PY{o}{=}\PY{l+s}{\PYZdq{}}\PY{l+s}{pink\PYZdq{}}\PY{p}{)}
         qqnorm\PY{p}{(}frota2\PY{p}{)}
         qqline\PY{p}{(}frota2\PY{p}{,} col\PY{o}{=}\PY{l+s}{\PYZdq{}}\PY{l+s}{cyan\PYZdq{}}\PY{p}{)}
         qqnorm\PY{p}{(}frota3\PY{p}{)}
         qqline\PY{p}{(}frota3\PY{p}{,} col\PY{o}{=}\PY{l+s}{\PYZdq{}}\PY{l+s}{green\PYZdq{}}\PY{p}{)}
\end{Verbatim}


    \begin{center}
    \adjustimage{max size={0.9\linewidth}{0.9\paperheight}}{output_12_0.png}
    \end{center}
    { \hspace*{\fill} \\}
    
    \begin{center}
    \adjustimage{max size={0.9\linewidth}{0.9\paperheight}}{output_12_1.png}
    \end{center}
    { \hspace*{\fill} \\}
    
    \begin{center}
    \adjustimage{max size={0.9\linewidth}{0.9\paperheight}}{output_12_2.png}
    \end{center}
    { \hspace*{\fill} \\}
    
    \begin{center}
    \adjustimage{max size={0.9\linewidth}{0.9\paperheight}}{output_12_3.png}
    \end{center}
    { \hspace*{\fill} \\}
    
    \begin{center}
    \adjustimage{max size={0.9\linewidth}{0.9\paperheight}}{output_12_4.png}
    \end{center}
    { \hspace*{\fill} \\}
    
    \begin{enumerate}
\def\labelenumi{\arabic{enumi})}
\setcounter{enumi}{1}
\tightlist
\item
  Um casco de navio consiste de 562 placas metálicas que devem ser
  rebitadas. Estima-se que o tempo gasto por um rebitador seja dado pela
  triangle (3h45,5h30,4h15) por placa e que o rebitador recebe USD 7.50
  por hora trabalhada.
\end{enumerate}

1 - Qual o risco de custo de mão-obra de rebitagem?

    \begin{Verbatim}[commandchars=\\\{\}]
{\color{incolor}In [{\color{incolor}97}]:} to\PYZus{}minutes \PY{o}{\PYZlt{}\PYZhy{}} \PY{k+kr}{function}\PY{p}{(}h\PY{p}{,}m\PY{p}{)} \PY{p}{(}h\PY{o}{*}\PY{l+m}{60}\PY{o}{+}m\PY{p}{)}\PY{o}{/}\PY{l+m}{60}
         to\PYZus{}dolar \PY{o}{\PYZlt{}\PYZhy{}} \PY{k+kr}{function}\PY{p}{(}h\PY{p}{,}m\PY{p}{)} to\PYZus{}minutes\PY{p}{(}h\PY{p}{,}m\PY{p}{)} \PY{o}{*}\PY{l+m}{7.5}\PY{o}{/}\PY{l+m}{1000}\PY{o}{*}\PY{l+m}{562}
         
         media \PY{o}{\PYZlt{}\PYZhy{}} \PY{k+kt}{c}\PY{p}{(}\PY{p}{)}
         \PY{k+kr}{for} \PY{p}{(}i \PY{k+kr}{in} \PY{l+m}{1}\PY{o}{:}\PY{l+m}{10000}\PY{p}{)}\PY{p}{\PYZob{}}
             tempo \PY{o}{\PYZlt{}\PYZhy{}} rtriangle\PY{p}{(}\PY{l+m}{562}\PY{p}{,}to\PYZus{}minutes\PY{p}{(}\PY{l+m}{3}\PY{p}{,}\PY{l+m}{45}\PY{p}{)}\PY{p}{,}to\PYZus{}minutes\PY{p}{(}\PY{l+m}{5}\PY{p}{,}\PY{l+m}{30}\PY{p}{)}\PY{p}{,}
                                to\PYZus{}minutes\PY{p}{(}\PY{l+m}{4}\PY{p}{,}\PY{l+m}{15}\PY{p}{)}\PY{p}{)}
             media \PY{o}{\PYZlt{}\PYZhy{}} \PY{k+kt}{c}\PY{p}{(}media\PY{p}{,} \PY{k+kp}{sum}\PY{p}{(}tempo\PY{p}{)}\PY{o}{*}\PY{l+m}{7.5}\PY{o}{/}\PY{l+m}{1000}\PY{p}{)}
         \PY{p}{\PYZcb{}}
         m \PY{o}{\PYZlt{}\PYZhy{}} \PY{k+kp}{mean}\PY{p}{(}media\PY{p}{)}
         s \PY{o}{\PYZlt{}\PYZhy{}} sd\PY{p}{(}media\PY{p}{)}
         
         m2 \PY{o}{=} \PY{p}{(}to\PYZus{}minutes\PY{p}{(}\PY{l+m}{3}\PY{p}{,}\PY{l+m}{45}\PY{p}{)}\PY{o}{+}to\PYZus{}minutes\PY{p}{(}\PY{l+m}{5}\PY{p}{,}\PY{l+m}{30}\PY{p}{)}\PY{o}{+}to\PYZus{}minutes\PY{p}{(}\PY{l+m}{4}\PY{p}{,}\PY{l+m}{15}\PY{p}{)}\PY{p}{)}\PY{o}{/}\PY{l+m}{3}\PY{o}{*}\PY{l+m}{562}\PY{o}{*}\PY{l+m}{7.5}\PY{o}{/}\PY{l+m}{1000}
         s2 \PY{o}{=} \PY{k+kp}{sqrt}\PY{p}{(}\PY{p}{(}to\PYZus{}dolar\PY{p}{(}\PY{l+m}{3}\PY{p}{,}\PY{l+m}{45}\PY{p}{)}\PY{o}{\PYZca{}}\PY{l+m}{2} \PY{o}{+} to\PYZus{}dolar\PY{p}{(}\PY{l+m}{5}\PY{p}{,}\PY{l+m}{30}\PY{p}{)}\PY{o}{\PYZca{}}\PY{l+m}{2} \PY{o}{+} to\PYZus{}dolar\PY{p}{(}\PY{l+m}{4}\PY{p}{,}\PY{l+m}{15}\PY{p}{)}\PY{o}{\PYZca{}}\PY{l+m}{2} \PY{o}{\PYZhy{}} to\PYZus{}dolar\PY{p}{(}\PY{l+m}{3}\PY{p}{,}\PY{l+m}{45}\PY{p}{)}\PY{o}{*}to\PYZus{}dolar\PY{p}{(}\PY{l+m}{5}\PY{p}{,}\PY{l+m}{30}\PY{p}{)} \PY{o}{\PYZhy{}}
                    to\PYZus{}dolar\PY{p}{(}\PY{l+m}{3}\PY{p}{,}\PY{l+m}{45}\PY{p}{)}\PY{o}{*}to\PYZus{}dolar\PY{p}{(}\PY{l+m}{4}\PY{p}{,}\PY{l+m}{15}\PY{p}{)} \PY{o}{\PYZhy{}} to\PYZus{}dolar\PY{p}{(}\PY{l+m}{5}\PY{p}{,}\PY{l+m}{30}\PY{p}{)}\PY{o}{*}to\PYZus{}dolar\PY{p}{(}\PY{l+m}{4}\PY{p}{,}\PY{l+m}{15}\PY{p}{)}\PY{p}{)}\PY{o}{/}\PY{l+m}{18}\PY{p}{)}\PY{o}{/}\PY{k+kp}{sqrt}\PY{p}{(}\PY{l+m}{562}\PY{p}{)}
         
         media\PYZus{}n \PY{o}{\PYZlt{}\PYZhy{}} rnorm\PY{p}{(}\PY{l+m}{10000}\PY{p}{,}m2\PY{p}{,}s2\PY{p}{)}
         hist\PY{p}{(}media\PY{p}{,} col\PY{o}{=}rgb\PY{p}{(}\PY{l+m}{1}\PY{p}{,}\PY{l+m}{0}\PY{p}{,}\PY{l+m}{0}\PY{p}{,}\PY{l+m}{0.3}\PY{p}{)}\PY{p}{,} main\PY{o}{=}\PY{l+s}{\PYZdq{}}\PY{l+s}{Histogram of cost in 1000 USD\PYZdq{}}\PY{p}{)}
         hist\PY{p}{(}media\PYZus{}n\PY{p}{,} col\PY{o}{=}rgb\PY{p}{(}\PY{l+m}{0}\PY{p}{,}\PY{l+m}{0}\PY{p}{,}\PY{l+m}{1}\PY{p}{,}\PY{l+m}{0.3}\PY{p}{)}\PY{p}{,} add\PY{o}{=}\PY{n+nb+bp}{T}\PY{p}{)}
\end{Verbatim}


    \begin{center}
    \adjustimage{max size={0.9\linewidth}{0.9\paperheight}}{output_14_0.png}
    \end{center}
    { \hspace*{\fill} \\}
    
    2 - Compare a distribuição cumulativa obtida usando a abordagem MC -
força bruta com aquela obtida usando o TCL. Use a função qqnorm para
visualizar a comparação dos resultados.

    \begin{Verbatim}[commandchars=\\\{\}]
{\color{incolor}In [{\color{incolor}98}]:} qqplot\PY{p}{(}media\PY{p}{,}media\PYZus{}n\PY{p}{,} main\PY{o}{=}\PY{l+s}{\PYZdq{}}\PY{l+s}{Q\PYZhy{}Q plot (in 1000 USD)\PYZdq{}}\PY{p}{)}
\end{Verbatim}


    \begin{center}
    \adjustimage{max size={0.9\linewidth}{0.9\paperheight}}{output_16_0.png}
    \end{center}
    { \hspace*{\fill} \\}
    
    \begin{enumerate}
\def\labelenumi{\arabic{enumi})}
\setcounter{enumi}{2}
\tightlist
\item
  Todas as sextas-feiras os principais executivos de uma empresa vão
  almoçar juntos, a convite da empresa. Entre 16 e 22 executivos
  participam destes almoços, sendo 18 o valor mais provável. Cada
  executivo consome entre USD 25 e USD 36, sendo USD 28 o valor mais
  provável. Sabendo que um ano possui 40, 41 ou 42 sextas-feiras úteis,
  avalie o risco do gasto anual da empresa com estes almoços.
\end{enumerate}

    \begin{Verbatim}[commandchars=\\\{\}]
{\color{incolor}In [{\color{incolor}100}]:} custo \PY{o}{\PYZlt{}\PYZhy{}} \PY{k+kt}{c}\PY{p}{(}\PY{p}{)}
          \PY{k+kr}{for} \PY{p}{(}i \PY{k+kr}{in} \PY{l+m}{1}\PY{o}{:}\PY{l+m}{1000}\PY{p}{)}\PY{p}{\PYZob{}}
              \PY{k+kr}{for} \PY{p}{(}j \PY{k+kr}{in} \PY{k+kt}{c}\PY{p}{(}\PY{l+m}{40}\PY{p}{,}\PY{l+m}{41}\PY{p}{,}\PY{l+m}{42}\PY{p}{)}\PY{p}{)}\PY{p}{\PYZob{}}
                  exec \PY{o}{\PYZlt{}\PYZhy{}} rtriangle\PY{p}{(}j\PY{p}{,}\PY{l+m}{16}\PY{p}{,}\PY{l+m}{22}\PY{p}{,}\PY{l+m}{18}\PY{p}{)}
                  almoco \PY{o}{\PYZlt{}\PYZhy{}} rtriangle\PY{p}{(}j\PY{p}{,}\PY{l+m}{25}\PY{p}{,}\PY{l+m}{36}\PY{p}{,}\PY{l+m}{28}\PY{p}{)}
                  custo \PY{o}{\PYZlt{}\PYZhy{}} \PY{k+kt}{c}\PY{p}{(}custo\PY{p}{,}\PY{k+kp}{sum}\PY{p}{(}exec \PY{o}{*} almoco\PY{p}{)}\PY{p}{)}
              \PY{p}{\PYZcb{}}
          \PY{p}{\PYZcb{}}
          
          \PY{k+kp}{round}\PY{p}{(}\PY{k+kp}{mean}\PY{p}{(}custo\PY{p}{)}\PY{p}{,}\PY{l+m}{2}\PY{p}{)}
          hist\PY{p}{(}custo\PY{p}{)}
\end{Verbatim}


    22705.33

    
    \begin{center}
    \adjustimage{max size={0.9\linewidth}{0.9\paperheight}}{output_18_1.png}
    \end{center}
    { \hspace*{\fill} \\}
    
    \begin{enumerate}
\def\labelenumi{\arabic{enumi})}
\setcounter{enumi}{3}
\tightlist
\item
  Um conhecido Chefe de cozinha deseja avaliar o retorno de investimento
  de seu novo restaurante no Rio de Janeiro. Através de consulta a
  alguns especialistas locais, ele ficou convencido que um conjunto de
  cliente típicos (2 pessoas para jantar ou almoçar) deve gastar cerca
  de 130,00 reais por visita ao restaurante.
\end{enumerate}

Entretanto, é importante mencionar que alguns poucos casais podem vir a
escolher um subconjunto dos pratos mais baratos, reduzindo este gasto
para um mínimo de 90,00 reais. No caso de conjuntos de 4 clientes, que
ocorre com uma certa frequência, o gasto máximo por visita ao
restaurante poderia chegar até 250,00 reais. Embora seja possível que o
restaurante venha a receber visitas de conjuntos de 6 ou mais pessoas, a
experiência mostra que estas situações são muito raras, causando um
impacto muito pequeno no faturamento total.

O Chefe calcula que seu restaurante no Rio de Janeiro, em regime, receba
a visita diária de pelo menos 40 grupos de pessoas. Em dias atípicos a
frequência pode chegar a até 120 grupos, sendo 60 o número mais
provável.

Outra informação relevante fornecida pelos especialistas é que se deve
esperar um lucro entre 15\% e 30\% do faturamento de cada mesa, sendo
que cerca de 22\% é o valor mais provável.

O Chefe quer que vc calcule o risco do valor presente do lucro total do
restaurante durante o primeiro ano de operação. Assuma que: todos os
meses do ano possuem 30 dias úteis e que os valores serão descontados
mensalmente pela taxa Selic.

    \begin{Verbatim}[commandchars=\\\{\}]
{\color{incolor}In [{\color{incolor}4}]:} lucro\PYZus{}ano \PY{o}{\PYZlt{}\PYZhy{}} \PY{k+kt}{c}\PY{p}{(}\PY{p}{)}
        \PY{k+kr}{for} \PY{p}{(}i \PY{k+kr}{in} \PY{l+m}{1}\PY{o}{:}\PY{l+m}{10000}\PY{p}{)}\PY{p}{\PYZob{}}
            lucro \PY{o}{\PYZlt{}\PYZhy{}} \PY{k+kt}{c}\PY{p}{(}\PY{p}{)}
            \PY{k+kr}{for} \PY{p}{(}mes \PY{k+kr}{in} \PY{l+m}{1}\PY{o}{:}\PY{l+m}{12}\PY{p}{)}\PY{p}{\PYZob{}}
                por\PYZus{}mesa \PY{o}{=} rtriangle\PY{p}{(}\PY{l+m}{30}\PY{p}{,}\PY{l+m}{90}\PY{p}{,}\PY{l+m}{250}\PY{p}{,}\PY{l+m}{130}\PY{p}{)}
                grupos \PY{o}{\PYZlt{}\PYZhy{}} rtriangle\PY{p}{(}\PY{l+m}{30}\PY{p}{,}\PY{l+m}{40}\PY{p}{,}\PY{l+m}{120}\PY{p}{,}\PY{l+m}{60}\PY{p}{)}
                fat\PYZus{}mesa \PY{o}{\PYZlt{}\PYZhy{}} por\PYZus{}mesa \PY{o}{*} grupos
                lucro\PYZus{}mes \PY{o}{\PYZlt{}\PYZhy{}} \PY{k+kp}{sum}\PY{p}{(}rtriangle\PY{p}{(}\PY{l+m}{30}\PY{p}{,}\PY{l+m}{.15}\PY{p}{,}\PY{l+m}{.30}\PY{p}{,}\PY{l+m}{.22}\PY{p}{)} \PY{o}{*} fat\PYZus{}mesa\PY{p}{)}
                lucro \PY{o}{\PYZlt{}\PYZhy{}} \PY{k+kt}{c}\PY{p}{(}lucro\PY{p}{,}lucro\PYZus{}mes \PY{o}{/} \PY{p}{(}\PY{p}{(}\PY{l+m}{1} \PY{o}{+} \PY{l+m}{.065}\PY{o}{/}\PY{l+m}{12}\PY{p}{)}\PY{o}{\PYZca{}}mes\PY{p}{)}\PY{p}{)}
            \PY{p}{\PYZcb{}}
            lucro\PYZus{}ano \PY{o}{\PYZlt{}\PYZhy{}} \PY{k+kt}{c}\PY{p}{(}lucro\PYZus{}ano\PY{p}{,}\PY{k+kp}{sum}\PY{p}{(}lucro\PY{p}{)}\PY{p}{)}
        \PY{p}{\PYZcb{}}
        \PY{k+kp}{mean}\PY{p}{(}lucro\PYZus{}ano\PY{p}{)}
        sd\PY{p}{(}lucro\PYZus{}ano\PY{p}{)}
        hist\PY{p}{(}lucro\PYZus{}ano\PY{p}{)}
\end{Verbatim}


    891983.392557205

    
    16634.3706895236

    
    \begin{center}
    \adjustimage{max size={0.9\linewidth}{0.9\paperheight}}{output_20_2.png}
    \end{center}
    { \hspace*{\fill} \\}
    

    % Add a bibliography block to the postdoc
    
    
    
    \end{document}
