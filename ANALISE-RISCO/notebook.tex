
% Default to the notebook output style

    


% Inherit from the specified cell style.




    
\documentclass[11pt]{article}

    
    
    \usepackage[T1]{fontenc}
    % Nicer default font (+ math font) than Computer Modern for most use cases
    \usepackage{mathpazo}

    % Basic figure setup, for now with no caption control since it's done
    % automatically by Pandoc (which extracts ![](path) syntax from Markdown).
    \usepackage{graphicx}
    % We will generate all images so they have a width \maxwidth. This means
    % that they will get their normal width if they fit onto the page, but
    % are scaled down if they would overflow the margins.
    \makeatletter
    \def\maxwidth{\ifdim\Gin@nat@width>\linewidth\linewidth
    \else\Gin@nat@width\fi}
    \makeatother
    \let\Oldincludegraphics\includegraphics
    % Set max figure width to be 80% of text width, for now hardcoded.
    \renewcommand{\includegraphics}[1]{\Oldincludegraphics[width=.8\maxwidth]{#1}}
    % Ensure that by default, figures have no caption (until we provide a
    % proper Figure object with a Caption API and a way to capture that
    % in the conversion process - todo).
    \usepackage{caption}
    \DeclareCaptionLabelFormat{nolabel}{}
    \captionsetup{labelformat=nolabel}

    \usepackage{adjustbox} % Used to constrain images to a maximum size 
    \usepackage{xcolor} % Allow colors to be defined
    \usepackage{enumerate} % Needed for markdown enumerations to work
    \usepackage{geometry} % Used to adjust the document margins
    \usepackage{amsmath} % Equations
    \usepackage{amssymb} % Equations
    \usepackage{textcomp} % defines textquotesingle
    % Hack from http://tex.stackexchange.com/a/47451/13684:
    \AtBeginDocument{%
        \def\PYZsq{\textquotesingle}% Upright quotes in Pygmentized code
    }
    \usepackage{upquote} % Upright quotes for verbatim code
    \usepackage{eurosym} % defines \euro
    \usepackage[mathletters]{ucs} % Extended unicode (utf-8) support
    \usepackage[utf8x]{inputenc} % Allow utf-8 characters in the tex document
    \usepackage{fancyvrb} % verbatim replacement that allows latex
    \usepackage{grffile} % extends the file name processing of package graphics 
                         % to support a larger range 
    % The hyperref package gives us a pdf with properly built
    % internal navigation ('pdf bookmarks' for the table of contents,
    % internal cross-reference links, web links for URLs, etc.)
    \usepackage{hyperref}
    \usepackage{longtable} % longtable support required by pandoc >1.10
    \usepackage{booktabs}  % table support for pandoc > 1.12.2
    \usepackage[inline]{enumitem} % IRkernel/repr support (it uses the enumerate* environment)
    \usepackage[normalem]{ulem} % ulem is needed to support strikethroughs (\sout)
                                % normalem makes italics be italics, not underlines
    

    
    
    % Colors for the hyperref package
    \definecolor{urlcolor}{rgb}{0,.145,.698}
    \definecolor{linkcolor}{rgb}{.71,0.21,0.01}
    \definecolor{citecolor}{rgb}{.12,.54,.11}

    % ANSI colors
    \definecolor{ansi-black}{HTML}{3E424D}
    \definecolor{ansi-black-intense}{HTML}{282C36}
    \definecolor{ansi-red}{HTML}{E75C58}
    \definecolor{ansi-red-intense}{HTML}{B22B31}
    \definecolor{ansi-green}{HTML}{00A250}
    \definecolor{ansi-green-intense}{HTML}{007427}
    \definecolor{ansi-yellow}{HTML}{DDB62B}
    \definecolor{ansi-yellow-intense}{HTML}{B27D12}
    \definecolor{ansi-blue}{HTML}{208FFB}
    \definecolor{ansi-blue-intense}{HTML}{0065CA}
    \definecolor{ansi-magenta}{HTML}{D160C4}
    \definecolor{ansi-magenta-intense}{HTML}{A03196}
    \definecolor{ansi-cyan}{HTML}{60C6C8}
    \definecolor{ansi-cyan-intense}{HTML}{258F8F}
    \definecolor{ansi-white}{HTML}{C5C1B4}
    \definecolor{ansi-white-intense}{HTML}{A1A6B2}

    % commands and environments needed by pandoc snippets
    % extracted from the output of `pandoc -s`
    \providecommand{\tightlist}{%
      \setlength{\itemsep}{0pt}\setlength{\parskip}{0pt}}
    \DefineVerbatimEnvironment{Highlighting}{Verbatim}{commandchars=\\\{\}}
    % Add ',fontsize=\small' for more characters per line
    \newenvironment{Shaded}{}{}
    \newcommand{\KeywordTok}[1]{\textcolor[rgb]{0.00,0.44,0.13}{\textbf{{#1}}}}
    \newcommand{\DataTypeTok}[1]{\textcolor[rgb]{0.56,0.13,0.00}{{#1}}}
    \newcommand{\DecValTok}[1]{\textcolor[rgb]{0.25,0.63,0.44}{{#1}}}
    \newcommand{\BaseNTok}[1]{\textcolor[rgb]{0.25,0.63,0.44}{{#1}}}
    \newcommand{\FloatTok}[1]{\textcolor[rgb]{0.25,0.63,0.44}{{#1}}}
    \newcommand{\CharTok}[1]{\textcolor[rgb]{0.25,0.44,0.63}{{#1}}}
    \newcommand{\StringTok}[1]{\textcolor[rgb]{0.25,0.44,0.63}{{#1}}}
    \newcommand{\CommentTok}[1]{\textcolor[rgb]{0.38,0.63,0.69}{\textit{{#1}}}}
    \newcommand{\OtherTok}[1]{\textcolor[rgb]{0.00,0.44,0.13}{{#1}}}
    \newcommand{\AlertTok}[1]{\textcolor[rgb]{1.00,0.00,0.00}{\textbf{{#1}}}}
    \newcommand{\FunctionTok}[1]{\textcolor[rgb]{0.02,0.16,0.49}{{#1}}}
    \newcommand{\RegionMarkerTok}[1]{{#1}}
    \newcommand{\ErrorTok}[1]{\textcolor[rgb]{1.00,0.00,0.00}{\textbf{{#1}}}}
    \newcommand{\NormalTok}[1]{{#1}}
    
    % Additional commands for more recent versions of Pandoc
    \newcommand{\ConstantTok}[1]{\textcolor[rgb]{0.53,0.00,0.00}{{#1}}}
    \newcommand{\SpecialCharTok}[1]{\textcolor[rgb]{0.25,0.44,0.63}{{#1}}}
    \newcommand{\VerbatimStringTok}[1]{\textcolor[rgb]{0.25,0.44,0.63}{{#1}}}
    \newcommand{\SpecialStringTok}[1]{\textcolor[rgb]{0.73,0.40,0.53}{{#1}}}
    \newcommand{\ImportTok}[1]{{#1}}
    \newcommand{\DocumentationTok}[1]{\textcolor[rgb]{0.73,0.13,0.13}{\textit{{#1}}}}
    \newcommand{\AnnotationTok}[1]{\textcolor[rgb]{0.38,0.63,0.69}{\textbf{\textit{{#1}}}}}
    \newcommand{\CommentVarTok}[1]{\textcolor[rgb]{0.38,0.63,0.69}{\textbf{\textit{{#1}}}}}
    \newcommand{\VariableTok}[1]{\textcolor[rgb]{0.10,0.09,0.49}{{#1}}}
    \newcommand{\ControlFlowTok}[1]{\textcolor[rgb]{0.00,0.44,0.13}{\textbf{{#1}}}}
    \newcommand{\OperatorTok}[1]{\textcolor[rgb]{0.40,0.40,0.40}{{#1}}}
    \newcommand{\BuiltInTok}[1]{{#1}}
    \newcommand{\ExtensionTok}[1]{{#1}}
    \newcommand{\PreprocessorTok}[1]{\textcolor[rgb]{0.74,0.48,0.00}{{#1}}}
    \newcommand{\AttributeTok}[1]{\textcolor[rgb]{0.49,0.56,0.16}{{#1}}}
    \newcommand{\InformationTok}[1]{\textcolor[rgb]{0.38,0.63,0.69}{\textbf{\textit{{#1}}}}}
    \newcommand{\WarningTok}[1]{\textcolor[rgb]{0.38,0.63,0.69}{\textbf{\textit{{#1}}}}}
    
    
    % Define a nice break command that doesn't care if a line doesn't already
    % exist.
    \def\br{\hspace*{\fill} \\* }
    % Math Jax compatability definitions
    \def\gt{>}
    \def\lt{<}
    % Document parameters
    \title{lista\_3}
    
    
    

    % Pygments definitions
    
\makeatletter
\def\PY@reset{\let\PY@it=\relax \let\PY@bf=\relax%
    \let\PY@ul=\relax \let\PY@tc=\relax%
    \let\PY@bc=\relax \let\PY@ff=\relax}
\def\PY@tok#1{\csname PY@tok@#1\endcsname}
\def\PY@toks#1+{\ifx\relax#1\empty\else%
    \PY@tok{#1}\expandafter\PY@toks\fi}
\def\PY@do#1{\PY@bc{\PY@tc{\PY@ul{%
    \PY@it{\PY@bf{\PY@ff{#1}}}}}}}
\def\PY#1#2{\PY@reset\PY@toks#1+\relax+\PY@do{#2}}

\expandafter\def\csname PY@tok@w\endcsname{\def\PY@tc##1{\textcolor[rgb]{0.73,0.73,0.73}{##1}}}
\expandafter\def\csname PY@tok@c\endcsname{\let\PY@it=\textit\def\PY@tc##1{\textcolor[rgb]{0.25,0.50,0.50}{##1}}}
\expandafter\def\csname PY@tok@cp\endcsname{\def\PY@tc##1{\textcolor[rgb]{0.74,0.48,0.00}{##1}}}
\expandafter\def\csname PY@tok@k\endcsname{\let\PY@bf=\textbf\def\PY@tc##1{\textcolor[rgb]{0.00,0.50,0.00}{##1}}}
\expandafter\def\csname PY@tok@kp\endcsname{\def\PY@tc##1{\textcolor[rgb]{0.00,0.50,0.00}{##1}}}
\expandafter\def\csname PY@tok@kt\endcsname{\def\PY@tc##1{\textcolor[rgb]{0.69,0.00,0.25}{##1}}}
\expandafter\def\csname PY@tok@o\endcsname{\def\PY@tc##1{\textcolor[rgb]{0.40,0.40,0.40}{##1}}}
\expandafter\def\csname PY@tok@ow\endcsname{\let\PY@bf=\textbf\def\PY@tc##1{\textcolor[rgb]{0.67,0.13,1.00}{##1}}}
\expandafter\def\csname PY@tok@nb\endcsname{\def\PY@tc##1{\textcolor[rgb]{0.00,0.50,0.00}{##1}}}
\expandafter\def\csname PY@tok@nf\endcsname{\def\PY@tc##1{\textcolor[rgb]{0.00,0.00,1.00}{##1}}}
\expandafter\def\csname PY@tok@nc\endcsname{\let\PY@bf=\textbf\def\PY@tc##1{\textcolor[rgb]{0.00,0.00,1.00}{##1}}}
\expandafter\def\csname PY@tok@nn\endcsname{\let\PY@bf=\textbf\def\PY@tc##1{\textcolor[rgb]{0.00,0.00,1.00}{##1}}}
\expandafter\def\csname PY@tok@ne\endcsname{\let\PY@bf=\textbf\def\PY@tc##1{\textcolor[rgb]{0.82,0.25,0.23}{##1}}}
\expandafter\def\csname PY@tok@nv\endcsname{\def\PY@tc##1{\textcolor[rgb]{0.10,0.09,0.49}{##1}}}
\expandafter\def\csname PY@tok@no\endcsname{\def\PY@tc##1{\textcolor[rgb]{0.53,0.00,0.00}{##1}}}
\expandafter\def\csname PY@tok@nl\endcsname{\def\PY@tc##1{\textcolor[rgb]{0.63,0.63,0.00}{##1}}}
\expandafter\def\csname PY@tok@ni\endcsname{\let\PY@bf=\textbf\def\PY@tc##1{\textcolor[rgb]{0.60,0.60,0.60}{##1}}}
\expandafter\def\csname PY@tok@na\endcsname{\def\PY@tc##1{\textcolor[rgb]{0.49,0.56,0.16}{##1}}}
\expandafter\def\csname PY@tok@nt\endcsname{\let\PY@bf=\textbf\def\PY@tc##1{\textcolor[rgb]{0.00,0.50,0.00}{##1}}}
\expandafter\def\csname PY@tok@nd\endcsname{\def\PY@tc##1{\textcolor[rgb]{0.67,0.13,1.00}{##1}}}
\expandafter\def\csname PY@tok@s\endcsname{\def\PY@tc##1{\textcolor[rgb]{0.73,0.13,0.13}{##1}}}
\expandafter\def\csname PY@tok@sd\endcsname{\let\PY@it=\textit\def\PY@tc##1{\textcolor[rgb]{0.73,0.13,0.13}{##1}}}
\expandafter\def\csname PY@tok@si\endcsname{\let\PY@bf=\textbf\def\PY@tc##1{\textcolor[rgb]{0.73,0.40,0.53}{##1}}}
\expandafter\def\csname PY@tok@se\endcsname{\let\PY@bf=\textbf\def\PY@tc##1{\textcolor[rgb]{0.73,0.40,0.13}{##1}}}
\expandafter\def\csname PY@tok@sr\endcsname{\def\PY@tc##1{\textcolor[rgb]{0.73,0.40,0.53}{##1}}}
\expandafter\def\csname PY@tok@ss\endcsname{\def\PY@tc##1{\textcolor[rgb]{0.10,0.09,0.49}{##1}}}
\expandafter\def\csname PY@tok@sx\endcsname{\def\PY@tc##1{\textcolor[rgb]{0.00,0.50,0.00}{##1}}}
\expandafter\def\csname PY@tok@m\endcsname{\def\PY@tc##1{\textcolor[rgb]{0.40,0.40,0.40}{##1}}}
\expandafter\def\csname PY@tok@gh\endcsname{\let\PY@bf=\textbf\def\PY@tc##1{\textcolor[rgb]{0.00,0.00,0.50}{##1}}}
\expandafter\def\csname PY@tok@gu\endcsname{\let\PY@bf=\textbf\def\PY@tc##1{\textcolor[rgb]{0.50,0.00,0.50}{##1}}}
\expandafter\def\csname PY@tok@gd\endcsname{\def\PY@tc##1{\textcolor[rgb]{0.63,0.00,0.00}{##1}}}
\expandafter\def\csname PY@tok@gi\endcsname{\def\PY@tc##1{\textcolor[rgb]{0.00,0.63,0.00}{##1}}}
\expandafter\def\csname PY@tok@gr\endcsname{\def\PY@tc##1{\textcolor[rgb]{1.00,0.00,0.00}{##1}}}
\expandafter\def\csname PY@tok@ge\endcsname{\let\PY@it=\textit}
\expandafter\def\csname PY@tok@gs\endcsname{\let\PY@bf=\textbf}
\expandafter\def\csname PY@tok@gp\endcsname{\let\PY@bf=\textbf\def\PY@tc##1{\textcolor[rgb]{0.00,0.00,0.50}{##1}}}
\expandafter\def\csname PY@tok@go\endcsname{\def\PY@tc##1{\textcolor[rgb]{0.53,0.53,0.53}{##1}}}
\expandafter\def\csname PY@tok@gt\endcsname{\def\PY@tc##1{\textcolor[rgb]{0.00,0.27,0.87}{##1}}}
\expandafter\def\csname PY@tok@err\endcsname{\def\PY@bc##1{\setlength{\fboxsep}{0pt}\fcolorbox[rgb]{1.00,0.00,0.00}{1,1,1}{\strut ##1}}}
\expandafter\def\csname PY@tok@kc\endcsname{\let\PY@bf=\textbf\def\PY@tc##1{\textcolor[rgb]{0.00,0.50,0.00}{##1}}}
\expandafter\def\csname PY@tok@kd\endcsname{\let\PY@bf=\textbf\def\PY@tc##1{\textcolor[rgb]{0.00,0.50,0.00}{##1}}}
\expandafter\def\csname PY@tok@kn\endcsname{\let\PY@bf=\textbf\def\PY@tc##1{\textcolor[rgb]{0.00,0.50,0.00}{##1}}}
\expandafter\def\csname PY@tok@kr\endcsname{\let\PY@bf=\textbf\def\PY@tc##1{\textcolor[rgb]{0.00,0.50,0.00}{##1}}}
\expandafter\def\csname PY@tok@bp\endcsname{\def\PY@tc##1{\textcolor[rgb]{0.00,0.50,0.00}{##1}}}
\expandafter\def\csname PY@tok@fm\endcsname{\def\PY@tc##1{\textcolor[rgb]{0.00,0.00,1.00}{##1}}}
\expandafter\def\csname PY@tok@vc\endcsname{\def\PY@tc##1{\textcolor[rgb]{0.10,0.09,0.49}{##1}}}
\expandafter\def\csname PY@tok@vg\endcsname{\def\PY@tc##1{\textcolor[rgb]{0.10,0.09,0.49}{##1}}}
\expandafter\def\csname PY@tok@vi\endcsname{\def\PY@tc##1{\textcolor[rgb]{0.10,0.09,0.49}{##1}}}
\expandafter\def\csname PY@tok@vm\endcsname{\def\PY@tc##1{\textcolor[rgb]{0.10,0.09,0.49}{##1}}}
\expandafter\def\csname PY@tok@sa\endcsname{\def\PY@tc##1{\textcolor[rgb]{0.73,0.13,0.13}{##1}}}
\expandafter\def\csname PY@tok@sb\endcsname{\def\PY@tc##1{\textcolor[rgb]{0.73,0.13,0.13}{##1}}}
\expandafter\def\csname PY@tok@sc\endcsname{\def\PY@tc##1{\textcolor[rgb]{0.73,0.13,0.13}{##1}}}
\expandafter\def\csname PY@tok@dl\endcsname{\def\PY@tc##1{\textcolor[rgb]{0.73,0.13,0.13}{##1}}}
\expandafter\def\csname PY@tok@s2\endcsname{\def\PY@tc##1{\textcolor[rgb]{0.73,0.13,0.13}{##1}}}
\expandafter\def\csname PY@tok@sh\endcsname{\def\PY@tc##1{\textcolor[rgb]{0.73,0.13,0.13}{##1}}}
\expandafter\def\csname PY@tok@s1\endcsname{\def\PY@tc##1{\textcolor[rgb]{0.73,0.13,0.13}{##1}}}
\expandafter\def\csname PY@tok@mb\endcsname{\def\PY@tc##1{\textcolor[rgb]{0.40,0.40,0.40}{##1}}}
\expandafter\def\csname PY@tok@mf\endcsname{\def\PY@tc##1{\textcolor[rgb]{0.40,0.40,0.40}{##1}}}
\expandafter\def\csname PY@tok@mh\endcsname{\def\PY@tc##1{\textcolor[rgb]{0.40,0.40,0.40}{##1}}}
\expandafter\def\csname PY@tok@mi\endcsname{\def\PY@tc##1{\textcolor[rgb]{0.40,0.40,0.40}{##1}}}
\expandafter\def\csname PY@tok@il\endcsname{\def\PY@tc##1{\textcolor[rgb]{0.40,0.40,0.40}{##1}}}
\expandafter\def\csname PY@tok@mo\endcsname{\def\PY@tc##1{\textcolor[rgb]{0.40,0.40,0.40}{##1}}}
\expandafter\def\csname PY@tok@ch\endcsname{\let\PY@it=\textit\def\PY@tc##1{\textcolor[rgb]{0.25,0.50,0.50}{##1}}}
\expandafter\def\csname PY@tok@cm\endcsname{\let\PY@it=\textit\def\PY@tc##1{\textcolor[rgb]{0.25,0.50,0.50}{##1}}}
\expandafter\def\csname PY@tok@cpf\endcsname{\let\PY@it=\textit\def\PY@tc##1{\textcolor[rgb]{0.25,0.50,0.50}{##1}}}
\expandafter\def\csname PY@tok@c1\endcsname{\let\PY@it=\textit\def\PY@tc##1{\textcolor[rgb]{0.25,0.50,0.50}{##1}}}
\expandafter\def\csname PY@tok@cs\endcsname{\let\PY@it=\textit\def\PY@tc##1{\textcolor[rgb]{0.25,0.50,0.50}{##1}}}

\def\PYZbs{\char`\\}
\def\PYZus{\char`\_}
\def\PYZob{\char`\{}
\def\PYZcb{\char`\}}
\def\PYZca{\char`\^}
\def\PYZam{\char`\&}
\def\PYZlt{\char`\<}
\def\PYZgt{\char`\>}
\def\PYZsh{\char`\#}
\def\PYZpc{\char`\%}
\def\PYZdl{\char`\$}
\def\PYZhy{\char`\-}
\def\PYZsq{\char`\'}
\def\PYZdq{\char`\"}
\def\PYZti{\char`\~}
% for compatibility with earlier versions
\def\PYZat{@}
\def\PYZlb{[}
\def\PYZrb{]}
\makeatother


    % Exact colors from NB
    \definecolor{incolor}{rgb}{0.0, 0.0, 0.5}
    \definecolor{outcolor}{rgb}{0.545, 0.0, 0.0}



    
    % Prevent overflowing lines due to hard-to-break entities
    \sloppy 
    % Setup hyperref package
    \hypersetup{
      breaklinks=true,  % so long urls are correctly broken across lines
      colorlinks=true,
      urlcolor=urlcolor,
      linkcolor=linkcolor,
      citecolor=citecolor,
      }
    % Slightly bigger margins than the latex defaults
    
    \geometry{verbose,tmargin=1in,bmargin=1in,lmargin=1in,rmargin=1in}
    
    

    \begin{document}
    
    
    \maketitle
    
    

    
    MAI 103: Análise de Risco // Prof. Eber Lista 03 // Data: 26/06/2018 //
Entrega: 03/07/2018

Luis Filipe Kopp

    Faça um modelo de risco (em R) para o custo de um projeto de um
gasoduto. A opção preferida para a rota do gasoduto tem uma extensão de
260 km. Existe um risco, porém, de que devido a oposição local, uma rota
alternativa com 290 km tenha que ser utilizada. Estima-se que a chance
que isto aconteça está na faixa entre 35\% a 40\%. A tubulação para o
gasoduto vem em seções de 8m de comprimento. As estimativas de custo (em
USD ) são mostradas na tabela abaixo.

    \begin{figure}
\centering
\includegraphics{attachment:image.png}
\caption{image.png}
\end{figure}

    \subsubsection{inicio lista 3}\label{inicio-lista-3}

    \subsection{1) as funções de probabilidade e suas cumulativas para o
custo total em função da percepção de incerteza da rota
alternativa}\label{as-funuxe7uxf5es-de-probabilidade-e-suas-cumulativas-para-o-custo-total-em-funuxe7uxe3o-da-percepuxe7uxe3o-de-incerteza-da-rota-alternativa}

    a função de probabilidade é a custo\_(prob), e a acumulada é
ecdf(custo\_(prob)).

Aplicando em .35, .40 ou de .35 a .4 de .01 em .01 apresenta
comportamento muito similar, como apresentado no gráfico 2.

No entanto, a ocorrência da rota alternativa só pode ser sim ou não.

    \begin{Verbatim}[commandchars=\\\{\}]
{\color{incolor}In [{\color{incolor}283}]:} \PY{k+kn}{library}\PY{p}{(}triangle\PY{p}{)}
          custo\PYZus{} \PY{o}{\PYZlt{}\PYZhy{}} \PY{k+kr}{function}\PY{p}{(}prob \PY{o}{=} \PY{l+m}{.35}\PY{p}{,} ns \PY{o}{=} \PY{l+m}{10000}\PY{p}{)} \PY{p}{\PYZob{}}
              distancia \PY{o}{\PYZlt{}\PYZhy{}}  \PY{p}{(}\PY{k+kp}{rep}\PY{p}{(}\PY{l+m}{260}\PY{p}{,}ns\PY{p}{)} \PY{o}{+} rbinom\PY{p}{(}ns\PY{p}{,}\PY{l+m}{1}\PY{p}{,}prob\PY{p}{)} \PY{o}{*} \PY{l+m}{30}\PY{p}{)} \PY{o}{*} \PY{l+m}{1000} \PY{c+c1}{\PYZsh{}\PYZsh{}\PYZsh{}\PYZsh{} em metros}
              n\PYZus{}tubos \PY{o}{\PYZlt{}\PYZhy{}} \PY{k+kp}{ceiling}\PY{p}{(}distancia \PY{o}{/} \PY{l+m}{8}\PY{p}{)}
              tub \PY{o}{\PYZlt{}\PYZhy{}}    rtriangle\PY{p}{(}ns\PY{p}{,}\PY{l+m}{725}\PY{p}{,}\PY{l+m}{790}\PY{p}{,}\PY{l+m}{740}\PY{p}{)} \PY{o}{*} n\PYZus{}tubos \PY{c+c1}{\PYZsh{}\PYZsh{}\PYZsh{} dolar}
              cavar \PY{o}{\PYZlt{}\PYZhy{}}  rtriangle\PY{p}{(}ns\PY{p}{,}\PY{l+m}{12}\PY{p}{,}\PY{l+m}{25}\PY{p}{,}\PY{l+m}{16}\PY{p}{)} \PY{o}{*} n\PYZus{}tubos   \PY{c+c1}{\PYZsh{}\PYZsh{}\PYZsh{} horas}
              mdo \PY{o}{\PYZlt{}\PYZhy{}}    rtriangle\PY{p}{(}ns\PY{p}{,}\PY{l+m}{17}\PY{p}{,}\PY{l+m}{23}\PY{p}{,}\PY{l+m}{18.5}\PY{p}{)}  \PY{c+c1}{\PYZsh{}\PYZsh{}\PYZsh{} dolar/hora}
              transp \PY{o}{\PYZlt{}\PYZhy{}} rtriangle\PY{p}{(}ns\PY{p}{,}\PY{l+m}{6.1}\PY{p}{,}\PY{l+m}{7.4}\PY{p}{,}\PY{l+m}{6.6}\PY{p}{)} \PY{o}{*} n\PYZus{}tubos \PY{c+c1}{\PYZsh{}\PYZsh{}\PYZsh{} dolar }
              sold \PY{o}{\PYZlt{}\PYZhy{}}   rtriangle\PY{p}{(}ns\PY{p}{,}\PY{l+m}{4}\PY{p}{,}\PY{l+m}{5}\PY{p}{,}\PY{l+m}{4.5}\PY{p}{)} \PY{o}{*} \PY{p}{(}n\PYZus{}tubos \PY{l+m}{\PYZhy{}1}\PY{p}{)} \PY{c+c1}{\PYZsh{}\PYZsh{}\PYZsh{} horas}
              filtr \PY{o}{\PYZlt{}\PYZhy{}}  rtriangle\PY{p}{(}ns\PY{p}{,}\PY{l+m}{165000}\PY{p}{,}\PY{l+m}{188000}\PY{p}{,}\PY{l+m}{173000}\PY{p}{)} \PY{c+c1}{\PYZsh{}\PYZsh{}\PYZsh{} dolar}
              acab \PY{o}{\PYZlt{}\PYZhy{}}   rtriangle\PY{p}{(}ns\PY{p}{,}\PY{l+m}{14000}\PY{p}{,}\PY{l+m}{17000}\PY{p}{,}\PY{l+m}{15000}\PY{p}{)} \PY{o}{*} distancia\PY{o}{/}\PY{l+m}{1000} \PY{c+c1}{\PYZsh{}\PYZsh{}\PYZsh{} dolar}
              \PY{k+kt}{c} \PY{o}{\PYZlt{}\PYZhy{}} \PY{p}{(}tub \PY{o}{+} cavar \PY{o}{*} mdo \PY{o}{+} transp \PY{o}{+} sold\PY{o}{*} mdo \PY{o}{+} filtr \PY{o}{+} acab\PY{p}{)} \PY{o}{/} \PY{l+m}{1000000}
          \PY{p}{\PYZcb{}}
\end{Verbatim}


    \begin{Verbatim}[commandchars=\\\{\}]
{\color{incolor}In [{\color{incolor}284}]:} cc \PY{o}{\PYZlt{}\PYZhy{}} \PY{k+kt}{c}\PY{p}{(}\PY{p}{)}
          \PY{k+kr}{for} \PY{p}{(}p \PY{k+kr}{in}  \PY{k+kp}{seq}\PY{p}{(}\PY{l+m}{.35}\PY{p}{,} \PY{l+m}{.40}\PY{p}{,} \PY{l+m}{.01}\PY{p}{)} \PY{p}{)}\PY{p}{\PYZob{}}
              cc \PY{o}{\PYZlt{}\PYZhy{}} \PY{k+kt}{c}\PY{p}{(}cc\PY{p}{,}custo\PYZus{}\PY{p}{(}p\PY{p}{,}\PY{l+m}{10000}\PY{p}{)}\PY{p}{)}
          \PY{p}{\PYZcb{}}
\end{Verbatim}


    \begin{Verbatim}[commandchars=\\\{\}]
{\color{incolor}In [{\color{incolor}285}]:} hist\PY{p}{(}custo\PYZus{}\PY{p}{(}\PY{l+m}{0}\PY{p}{)}\PY{p}{,} col\PY{o}{=}rgb\PY{p}{(}\PY{l+m}{0}\PY{p}{,}\PY{l+m}{1}\PY{p}{,}\PY{l+m}{0}\PY{p}{,}\PY{l+m}{.3}\PY{p}{)}\PY{p}{,}breaks \PY{o}{=}\PY{l+m}{100}\PY{p}{,} xlim\PY{o}{=}\PY{k+kt}{c}\PY{p}{(}\PY{l+m}{37}\PY{p}{,}\PY{l+m}{57}\PY{p}{)}\PY{p}{,} probability \PY{o}{=} \PY{n+nb+bp}{T}\PY{p}{,} 
               main\PY{o}{=}\PY{l+s}{\PYZdq{}}\PY{l+s}{graf 1 \PYZhy{} histograma para 260 (verde), média (vermelho) e 290km (azul)\PYZdq{}}\PY{p}{)}   \PY{c+c1}{\PYZsh{}\PYZsh{}\PYZsh{} green}
          hist\PY{p}{(}cc\PY{p}{,} col\PY{o}{=}rgb\PY{p}{(}\PY{l+m}{1}\PY{p}{,}\PY{l+m}{0}\PY{p}{,}\PY{l+m}{0}\PY{p}{,}\PY{l+m}{.3}\PY{p}{)}\PY{p}{,}breaks \PY{o}{=}\PY{l+m}{100}\PY{p}{,} xlim\PY{o}{=}\PY{k+kt}{c}\PY{p}{(}\PY{l+m}{37}\PY{p}{,}\PY{l+m}{57}\PY{p}{)}\PY{p}{,} probability \PY{o}{=} \PY{n+nb+bp}{T}\PY{p}{,} add\PY{o}{=}\PY{n+nb+bp}{T}\PY{p}{)}   \PY{c+c1}{\PYZsh{}\PYZsh{}\PYZsh{} red}
          hist\PY{p}{(}custo\PYZus{}\PY{p}{(}\PY{l+m}{1}\PY{p}{)}\PY{p}{,} col\PY{o}{=}rgb\PY{p}{(}\PY{l+m}{0}\PY{p}{,}\PY{l+m}{0}\PY{p}{,}\PY{l+m}{1}\PY{p}{,}\PY{l+m}{.3}\PY{p}{)}\PY{p}{,}breaks \PY{o}{=}\PY{l+m}{100}\PY{p}{,} xlim\PY{o}{=}\PY{k+kt}{c}\PY{p}{(}\PY{l+m}{37}\PY{p}{,}\PY{l+m}{57}\PY{p}{)}\PY{p}{,} probability \PY{o}{=} \PY{n+nb+bp}{T}\PY{p}{,} add\PY{o}{=}\PY{n+nb+bp}{T}\PY{p}{)}  \PY{c+c1}{\PYZsh{}\PYZsh{}\PYZsh{} blue}
          box\PY{p}{(}\PY{p}{)}
\end{Verbatim}


    \begin{center}
    \adjustimage{max size={0.9\linewidth}{0.9\paperheight}}{output_8_0.png}
    \end{center}
    { \hspace*{\fill} \\}
    
    \begin{Verbatim}[commandchars=\\\{\}]
{\color{incolor}In [{\color{incolor}286}]:} plot\PY{p}{(}density\PY{p}{(}cc\PY{p}{)}\PY{p}{,} xlim \PY{o}{=} \PY{k+kt}{c}\PY{p}{(}\PY{l+m}{37}\PY{p}{,}\PY{l+m}{57}\PY{p}{)}\PY{p}{,} ylim \PY{o}{=} \PY{k+kt}{c}\PY{p}{(}\PY{l+m}{0}\PY{p}{,}\PY{l+m}{.2}\PY{p}{)}\PY{p}{,} col\PY{o}{=}\PY{l+s}{\PYZdq{}}\PY{l+s}{red\PYZdq{}}\PY{p}{,} main\PY{o}{=}\PY{l+s}{\PYZdq{}}\PY{l+s}{Graf 2 \PYZhy{} Densidade em linhas\PYZdq{}}\PY{p}{)}
          lines\PY{p}{(}density\PY{p}{(}custo\PYZus{}\PY{p}{(}\PY{l+m}{.35}\PY{p}{)}\PY{p}{)}\PY{p}{,} col\PY{o}{=}\PY{l+s}{\PYZdq{}}\PY{l+s}{pink\PYZdq{}}\PY{p}{)}
          lines\PY{p}{(}density\PY{p}{(}custo\PYZus{}\PY{p}{(}\PY{l+m}{.4}\PY{p}{)}\PY{p}{)}\PY{p}{,} col\PY{o}{=}\PY{l+s}{\PYZdq{}}\PY{l+s}{cyan\PYZdq{}}\PY{p}{)}
          lines\PY{p}{(}density\PY{p}{(}custo\PYZus{}\PY{p}{(}\PY{l+m}{1}\PY{p}{)}\PY{p}{)}\PY{p}{,} col\PY{o}{=}\PY{l+s}{\PYZdq{}}\PY{l+s}{blue\PYZdq{}}\PY{p}{)}
          lines\PY{p}{(}density\PY{p}{(}custo\PYZus{}\PY{p}{(}\PY{l+m}{0}\PY{p}{)}\PY{p}{)}\PY{p}{,} col\PY{o}{=}\PY{l+s}{\PYZdq{}}\PY{l+s}{green\PYZdq{}}\PY{p}{)}
\end{Verbatim}


    \begin{center}
    \adjustimage{max size={0.9\linewidth}{0.9\paperheight}}{output_9_0.png}
    \end{center}
    { \hspace*{\fill} \\}
    
    \begin{Verbatim}[commandchars=\\\{\}]
{\color{incolor}In [{\color{incolor}300}]:} plot\PY{p}{(}ecdf\PY{p}{(}cc\PY{p}{)}\PY{p}{,} main\PY{o}{=}\PY{l+s}{\PYZdq{}}\PY{l+s}{Graf 3 \PYZhy{} Probabilidade Cumulativa P(Custo\PYZlt{}x)\PYZdq{}}\PY{p}{,} col\PY{o}{=}\PY{l+s}{\PYZdq{}}\PY{l+s}{red\PYZdq{}}\PY{p}{)}
          plot\PY{p}{(}ecdf\PY{p}{(}custo\PYZus{}\PY{p}{(}\PY{l+m}{.35}\PY{p}{)}\PY{p}{)}\PY{p}{,} add\PY{o}{=}\PY{n+nb+bp}{T}\PY{p}{,} col\PY{o}{=}\PY{l+s}{\PYZdq{}}\PY{l+s}{pink\PYZdq{}}\PY{p}{)}
          plot\PY{p}{(}ecdf\PY{p}{(}custo\PYZus{}\PY{p}{(}\PY{l+m}{.4}\PY{p}{)}\PY{p}{)}\PY{p}{,} add\PY{o}{=}\PY{n+nb+bp}{T}\PY{p}{,} col\PY{o}{=}\PY{l+s}{\PYZdq{}}\PY{l+s}{cyan\PYZdq{}}\PY{p}{)}
          plot\PY{p}{(}ecdf\PY{p}{(}custo\PYZus{}\PY{p}{(}\PY{l+m}{0}\PY{p}{)}\PY{p}{)}\PY{p}{,} add\PY{o}{=}\PY{n+nb+bp}{T}\PY{p}{,} col\PY{o}{=}\PY{l+s}{\PYZdq{}}\PY{l+s}{green\PYZdq{}}\PY{p}{)}
          plot\PY{p}{(}ecdf\PY{p}{(}custo\PYZus{}\PY{p}{(}\PY{l+m}{1}\PY{p}{)}\PY{p}{)}\PY{p}{,} add\PY{o}{=}\PY{n+nb+bp}{T}\PY{p}{,} col\PY{o}{=}\PY{l+s}{\PYZdq{}}\PY{l+s}{blue\PYZdq{}}\PY{p}{)}
\end{Verbatim}


    \begin{center}
    \adjustimage{max size={0.9\linewidth}{0.9\paperheight}}{output_10_0.png}
    \end{center}
    { \hspace*{\fill} \\}
    
    \subsubsection{2) assumindo que vc é o proponente: qual seria o preço
proposto, o alvo de custo e o valor contingenciado da
obra?}\label{assumindo-que-vc-uxe9-o-proponente-qual-seria-o-preuxe7o-proposto-o-alvo-de-custo-e-o-valor-contingenciado-da-obra}

    o preço proposto depende de:

\begin{enumerate}
\def\labelenumi{\arabic{enumi})}
\item
  margem de lucro média praticada no setor
\item
  o retorno depende do risco tomado, e a percepção de risco depende do
  lugar e do risco já tomado na carteira da empresa.
\end{enumerate}

Como essas informações não estão disponíveis, fizemos apenas a análise
de custo

    \begin{Verbatim}[commandchars=\\\{\}]
{\color{incolor}In [{\color{incolor}291}]:} inv\PYZus{}ecdf \PY{o}{\PYZlt{}\PYZhy{}} \PY{k+kr}{function}\PY{p}{(}f\PY{p}{)}\PY{p}{\PYZob{}} 
                  x \PY{o}{\PYZlt{}\PYZhy{}} \PY{k+kp}{environment}\PY{p}{(}f\PY{p}{)}\PY{o}{\PYZdl{}}x 
                  y \PY{o}{\PYZlt{}\PYZhy{}} \PY{k+kp}{environment}\PY{p}{(}f\PY{p}{)}\PY{o}{\PYZdl{}}y 
                  approxfun\PY{p}{(}y\PY{p}{,} x\PY{p}{)} 
          \PY{p}{\PYZcb{}} 
          
          com260 \PY{o}{\PYZlt{}\PYZhy{}} inv\PYZus{}ecdf\PY{p}{(}ecdf\PY{p}{(}custo\PYZus{}\PY{p}{(}\PY{l+m}{0}\PY{p}{)}\PY{p}{)}\PY{p}{)}\PY{p}{(}\PY{l+m}{.85}\PY{p}{)}
          \PY{k+kp}{print}\PY{p}{(}\PY{k+kt}{c}\PY{p}{(}\PY{l+s}{\PYZdq{}}\PY{l+s}{melhor caso (260km)\PYZdq{}}\PY{p}{,}com260\PY{p}{)}\PY{p}{)}
          
          com290 \PY{o}{\PYZlt{}\PYZhy{}} inv\PYZus{}ecdf\PY{p}{(}ecdf\PY{p}{(}custo\PYZus{}\PY{p}{(}\PY{l+m}{1}\PY{p}{)}\PY{p}{)}\PY{p}{)}\PY{p}{(}\PY{l+m}{.85}\PY{p}{)}
          \PY{k+kp}{print}\PY{p}{(}\PY{k+kt}{c}\PY{p}{(}\PY{l+s}{\PYZdq{}}\PY{l+s}{pior caso (290km)\PYZdq{}}\PY{p}{,}com290\PY{p}{)}\PY{p}{)}
          
          provavel \PY{o}{\PYZlt{}\PYZhy{}} inv\PYZus{}ecdf\PY{p}{(}ecdf\PY{p}{(}cc\PY{p}{)}\PY{p}{)}\PY{p}{(}\PY{l+m}{.85}\PY{p}{)}
          \PY{k+kp}{print}\PY{p}{(}\PY{k+kt}{c}\PY{p}{(}\PY{l+s}{\PYZdq{}}\PY{l+s}{ALVO \PYZhy{} custo esperado considerando risco 35\PYZhy{}40\PYZpc{}\PYZdq{}}\PY{p}{,}provavel\PY{p}{)}\PY{p}{)}
          
          \PY{k+kp}{print}\PY{p}{(}\PY{k+kt}{c}\PY{p}{(}\PY{l+s}{\PYZdq{}}\PY{l+s}{CONTINGÊNCIA \PYZhy{} alvo até pior caso\PYZdq{}}\PY{p}{,}com290 \PY{o}{\PYZhy{}} provavel\PY{p}{)}\PY{p}{)}
\end{Verbatim}


    \begin{Verbatim}[commandchars=\\\{\}]
[1] "melhor caso (260km)" "45.0503551731375"   
[1] "pior caso (290km)" "50.2128561860576" 
[1] "ALVO - custo esperado considerando risco 35-40\%"
[2] "48.241532580849"                                
[1] "CONTINGÊNCIA - alvo até pior caso" "1.97132360520865"                 

    \end{Verbatim}

    \begin{Verbatim}[commandchars=\\\{\}]
{\color{incolor}In [{\color{incolor}299}]:} \PY{c+c1}{\PYZsh{}\PYZsh{}\PYZsh{} outra forma de calcular a reserva de contingência é ver diferença entre pior e melhor caso e multiplicar pelo risco}
          
          reserva \PY{o}{\PYZlt{}\PYZhy{}} \PY{p}{(}com290 \PY{o}{\PYZhy{}} com260\PY{p}{)} \PY{o}{*} \PY{l+m}{.375}
          reserva
\end{Verbatim}


    1.93593787984505

    
    \subsubsection{3) assumindo que vc é o contratante: aceitaria uma
proposta de USD \$45
M?}\label{assumindo-que-vc-uxe9-o-contratante-aceitaria-uma-proposta-de-usd-45-m}

    Não, tem 43.4\% de chance de o projeto ter um custo maior que USD 45M.

Uma empresa que oferece o serviço por 45M não dimensionou ou não previu
o risco da rota alternativa. Já que se houver a rota alternativa teria
89.7\% de chance do custo ser maior que 45 milhões de dólares

    \begin{Verbatim}[commandchars=\\\{\}]
{\color{incolor}In [{\color{incolor}302}]:} \PY{k+kp}{print}\PY{p}{(}\PY{k+kt}{c}\PY{p}{(}\PY{l+s}{\PYZdq{}}\PY{l+s}{chance de custo maior que 45M na média\PYZdq{}}\PY{p}{,} \PY{l+m}{1} \PY{o}{\PYZhy{}} ecdf\PY{p}{(}cc\PY{p}{)}\PY{p}{(}\PY{l+m}{45}\PY{p}{)}\PY{p}{)}\PY{p}{)}
          \PY{k+kp}{print}\PY{p}{(}\PY{k+kt}{c}\PY{p}{(}\PY{l+s}{\PYZdq{}}\PY{l+s}{chance de pior caso maior que 45M\PYZdq{}}\PY{p}{,}\PY{l+m}{1} \PY{o}{\PYZhy{}} ecdf\PY{p}{(}custo\PYZus{}\PY{p}{(}\PY{l+m}{1}\PY{p}{)}\PY{p}{)}\PY{p}{(}\PY{l+m}{45}\PY{p}{)}\PY{p}{)}\PY{p}{)}
\end{Verbatim}


    \begin{Verbatim}[commandchars=\\\{\}]
[1] "chance de custo maior que 45M na média"
[2] "0.432066666666667"                     
[1] "chance de pior caso maior que 45M" "0.8991"                           

    \end{Verbatim}

    \begin{Verbatim}[commandchars=\\\{\}]
{\color{incolor}In [{\color{incolor}317}]:} \PY{k+kp}{print}\PY{p}{(}\PY{k+kt}{c}\PY{p}{(}\PY{l+s}{\PYZdq{}}\PY{l+s}{chance na média do risco para custo maior que \PYZdq{}}\PY{p}{,}\PY{k+kp}{round}\PY{p}{(}provavel\PY{p}{,}\PY{l+m}{2}\PY{p}{)}\PY{p}{,} \PY{k+kp}{round}\PY{p}{(}\PY{l+m}{1} \PY{o}{\PYZhy{}} ecdf\PY{p}{(}cc\PY{p}{)}\PY{p}{(}provavel\PY{p}{)}\PY{p}{,}\PY{l+m}{4}\PY{p}{)}\PY{p}{)}\PY{p}{)}
          \PY{k+kp}{print}\PY{p}{(}\PY{k+kt}{c}\PY{p}{(}\PY{l+s}{\PYZdq{}}\PY{l+s}{chance de pior caso maior que \PYZdq{}}\PY{p}{,}\PY{k+kp}{round}\PY{p}{(}provavel\PY{p}{,}\PY{l+m}{2}\PY{p}{)}\PY{p}{,}\PY{k+kp}{round}\PY{p}{(}\PY{l+m}{1} \PY{o}{\PYZhy{}} ecdf\PY{p}{(}custo\PYZus{}\PY{p}{(}\PY{l+m}{1}\PY{p}{)}\PY{p}{)}\PY{p}{(}provavel\PY{p}{)}\PY{p}{,}\PY{l+m}{3}\PY{p}{)}\PY{p}{)}\PY{p}{)}
\end{Verbatim}


    \begin{Verbatim}[commandchars=\\\{\}]
[1] "chance na média do risco para custo maior que "
[2] "48.24"                                         
[3] "0.15"                                          
[1] "chance de pior caso maior que " "48.24"                         
[3] "0.387"                         

    \end{Verbatim}


    % Add a bibliography block to the postdoc
    
    
    
    \end{document}
